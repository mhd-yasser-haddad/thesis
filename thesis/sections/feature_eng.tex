\section{Feature Engineering}

Feature engineering is vital to the reference parsing system introduced in this work. The bibliographic reference segmentation is a prediction problem, one where the model must assign a field-level label (author, title, year, etc.) to each token in a sequence. In order to be able to do this efficiently, the model requires informative representations of each token, those representations should capture both the token’s content and its contextual significance.

In the past few years, there has been a growing popularity in employing deep contextual embeddings like BERT~\cite{2019-bert} to embed text in natural language processing tasks. Even in tasks like citation parsing, where structure and formatting have strong semantic clues, traditional hand-engineered features can still have a significant value. They extract surface information that is often consistent across citation styles, like punctuation patterns, capitalization, or token order.

To gain a balance between generalization and interpretability, our system uses two types of features:
\begin{enumerate}
\item A set of \textbf{handcrafted features} inspired by the AnyStyle~\cite{anystyle} citation parser.
\item \textbf{Learned embeddings} derived from either subword-level models (BPE)~\cite{bpemb} or deep contextual models (BERT)~\cite{2019-bert}.
\end{enumerate}
In the following sections, we describe each feature group in detail, beginning with the handcrafted features.

\subsection{Handcrafted Features}
In this work, one of the design decisions was to complement neural embeddings with a robust set of hand-engineered features. These features provide interpretable, low-level cues that are very useful for detecting structural patterns in citation strings, patterns that are often overlooked ot inconsistently represented in pretrained language models. Our approach to feature engineering is influenced by AnyStyle, an open-source library for citation parsing developed by Sylvester Keil~\cite{anystyle}.

AnyStyle models citation parsing as a sequence labeling task and relies on a CRF model that has been trained on a wide range of citation styles. Unlike neural architectures, which can rely a lot on pretraining and embeddings, AnyStyle achieves high performance through a well-crafted set of features that encode orthographic, lexical, and contextual information.
These include features such as token capitalization, punctuation patterns, and dictionary lookups, all these features are designed to enable the model to detect bibliography entities in different types of citations.

In our system, we take and extend this method by constructing a set of features to feed into the model, concatenated with subword and contextual embeddings. This mixed representation allows the model to take advantage of deep contextual understanding through embeddings and interpretable surface signals from the hand-engineered features.
To facilitate compatibility, we re-implemented a subset of AnyStyle’s feature classes, modifying or adding to some where necessary.

In the following sections, we’ll describe each feature class, its role, and how it helps in the overall reference parsing task. 

\subsubsection{Affix Feature}
The affix feature extracts fixed-length prefixes and suffixes of tokens. It is useful for recording common patterns in names, abbreviations, or technical terms that appear repeatedly in bibliography citations. For instance, author initials like “J.” or “Ph.”, or journal abbreviations like “IEEE” or “JMLR”, have a specific pattern at the start or end of words.
This feature works by taking a number of characters from the beginning (prefix) or end (suffix) of the token. In our case, two variations of this feature class were applied: one that takes the first two characters (prefix) from the token, and another one that takes the last two characters (suffix) from the token. These affix pieces are built up incrementally–so the prefix extractor for a token like “Journal” would extract “J” and “Jo”, and the suffix extractor would extract “l” and “al”.

Affix features make the model able to generalize to out-of-vocabulary tokens by picking up on subword patterns that could indicate specific field types. For example, journal names often have a shared suffix like “-ology” or “-ics”, and author names often have a predictable pattern of abbreviations. While it’s simple in structure, these features are helpful in field prediction, especially where token-level embeddings on their own can sometimes lack attention to detail.

\subsubsection{Brackets Feature}
The brackets feature captures data on whether a token is enclosed by or adjacent to typical bracket symbols, such as parentheses \texttt{()}, square brackets \texttt{[]}, or angle brackets \texttt{<>}. This is a useful feature because citation formats could put some metadata in brackets – e.g., publication years, volume numbers, or reference indices – to visually distinguish them from other fields.
This feature adds a tag to the token, such as \texttt{parens}, \texttt{square-brackets}, \texttt{angle}, or more specific tags such as \texttt{opening-paren}, \texttt{closing-square-bracket}, etc., depending on the shape of the token. For example, the token (2003) would be tagged as \texttt{parens}, and a token starting with \texttt{[} would be tagged as \texttt{opening-square-bracket}.
Tokens without any brackets are simply tagged as \texttt{none}. When the token does not belong to one of the typical patterns, it is tagged as \texttt{other}.

By flagging bracket types and positions, this feature enables the model to recognize common citation patterns, like parenthesized years or square-bracketed reference numbers, that are used across many citation styles. This kind of formatting hint is sometimes necessary to help get an accurate segmentation and recognize which field this token might belong to, especially in noisy or reference strings extracted using OCR.

\subsubsection{Canonical Feature}
The canonical feature provides a normalized representation of each token by removing accents and converting it to a standardized, lowercase form. The main goal is that tokens of the same structure or meaning are treated in the same way by the model, whether they appear different due to language, style, or formatting differences.
In order to achieve this, the feature captures the first shape of the token (before any change) and performs some normalization operations. It initially performs Unicode normalization to break the characters down to their base form – i.e., separating a letter from its accent marks. It then removes all the accents and other diacritical marks, retaining only the base characters.
Finally, it scrubs the string of any additional formatting noise and makes it lowercase.

For instance, the name \texttt{García} is shortened into \texttt{garcia}, and \texttt{MÜLLER} to \texttt{muller}. This makes it easier for the model to recognize that these are likely the same name or entity with minor stylistic or typographic differences.
This capability proves useful, especially while tokenizing citations with names, titles, or journal names in multiple languages or styles. It minimizes inconsistency in the input and allows the model to generalize more effectively over the large variety of tokens found in citation strings.

\subsubsection{Caps Feature}
The caps feature tracks information about the capitalization pattern of each token. Capitalization tends to be an effective cue in reference strings — i.e., it can signal the occurrence of names, acronyms, or stylized title spacing. By observing and categorizing these patterns, the model has the ability to enhance on identify between different types of fields such as author names, journal names, or abbreviations.

This property works by looking at the original shape of the token and assigning one of a set of classes depending on whether and where there is one or more uppercase letters:
\begin{compactitem}
\item \textbf{single:} A single uppercase letter (e.g., \texttt{J}), common in shortened names or initials.
\item \textbf{initial:} A token that starts with an uppercase letter followed by a lowercase letter (e.g., \texttt{John}), typically used for names or title-cased words.
\item \textbf{caps:} A word token that consists entirely of all capital letters (e.g., \texttt{IEEE}, \texttt{SCIENCE}), typically delimiting acronyms or publication titles.
\item \textbf{lower:} A word token consisting only of lowercase letters (e.g., \texttt{and}, \texttt{volume}), typically less informative in a stand-alone.
\item \textbf{other:} An extra bucket for anything token-wise not accommodated by the above patterns, e.g., words of mixed cases, numbers, or punctuation.
\end{compactitem}

By assigning tokens to these capitalization classes, the model is more likely to correctly infer the possible function of a token within a citation, particularly when combined with other structural or contextual information.

\subsubsection{Category Feature}
The category feature determines Unicode character type of the first and last token characters to yield basic structural details. Through this analysis, the model achieves token classification based on characters, which provides information about first-letter identification and terminal punctuation, and symbol and number presence.

The feature retrieves the first and last tokens and then assigns them to the appropriate Unicode general categories through its mapping process. The Unicode general categories cover uppercase letters (\texttt{Lu}), lowercase letters (\texttt{Ll}), modifier letters (\texttt{Lm}), numbers (\texttt{N}), punctuation types (\texttt{P}, \texttt{Pc}, \texttt{Pd}, etc.), symbols (\texttt{S}), and unspecified other categories. Character data that cannot be categorized goes under the category of \texttt{none}.

For example:
\begin{compactitem}
\item When applied to \texttt{Vol.} the Vol. feature would provide output as \texttt{Lu} for \texttt{V} and \texttt{P} for the period.
\item The token 2023) provides both \texttt{N} as a number category and \texttt{Pe} for parenthesis.
\end{compactitem}
The model can apply its knowledge to numerous token forms because this feature ignores token content. The reference string benefits from this feature if it contains structured formatting cues that indicate field boundaries or field types through specific brackets or numbers and characters.

\subsubsection{Dictionary Feature}
The dictionary feature tries to see if a token matches a known word in a collection of lists that includes names, places, publishers, and journal names. The lists are being read from a dictionary file, grouped into the lists mentioned above. It is useful for detecting entities that would show up in a frequent way in reference strings, such as publisher names, city names of publication, or common journal abbreviations.
Each token is compared and checked to see if it exists in one of these lists:
\begin{compactitem}
\item \textbf{name} - common first or last names in the author field
\item \textbf{place} - cities or locations typically in publisher field
\item \textbf{publisher} - publishing firm or organisation names
\item \textbf{journal} - full journal titles or journal abbreviations
\end{compactitem}
If a token exists in one of the lists, it is marked as \texttt{T} (true) for that category; otherwise, it’ll be marked \texttt{F} (false). The output vector from this feature will have four components, each referencing one of the dictionary categories.
For example, token \texttt{Springer} might return \texttt{[F, F, T, F]}, which indicates that there is a match in the publisher dictionary but not in other dictionaries. This feature will allow the model to create a connection between a token with a role for it, which could lead to an increase in accuracy in the field segmentation.

The dictionary lists are extracted from structured reference data and preprocessed for consistency. Although these features are not very reliable and couldn’t be sufficient to provide correct labeling, they help with adding precision signals when used with contextual or structural features.

\subsubsection{Keyword Feature}
The keyword feature uses keyword patterns to assess token classification regarding previously known semantic categories. This detection method is intended to identify particular metadata signals within citation strings by analyzing both words and symbols.

The internal operation uses pre-determined groups of regular expressions that control category matching. The semantic roles (\texttt{editor}, \texttt{journal}, \texttt{date}, \texttt{volume}) exist as separate categories, and the system checks each token against different language versions of relevant keywords. Editor-related terminology in the metadata section contains multiple English designs, including \texttt{ed.}, \texttt{editors}, and \texttt{edited}, in addition to German (\texttt{herausgeber}) and Spanish or French equivalents (\texttt{compilador}). The grammar accepts character sequences from the symbolic series and the CJK series.
This feature evaluates by testing every pattern one by one until it encounters a category expression that perfectly matches the current token. The feature operates without producing an output whenever no suitable matching pattern exists.
Some example categories include:
\begin{compactitem}
\item \textbf{editor:} tokens like \texttt{ed.}, \texttt{Hrsg.}
\item \textbf{volume:} \texttt{vol.}, \texttt{no.}, \texttt{issue}, \texttt{heft}
\item \textbf{date:} matches several months or season names such as \texttt{May}, \texttt{Fall}, and \texttt{Herbst}.
\item \textbf{journal:} words like \texttt{Journal}, \texttt{Quarterly}, \texttt{Review}, or \texttt{Zeitschrift}
\item \textbf{accessed:} metadata like \texttt{retrieved}, \texttt{accessed}, \texttt{abgerufen}.
\item \textbf{locator:} \texttt{doi}, \texttt{url}
\item \textbf{etal:} short forms like \texttt{et al.}, \texttt{others}
\end{compactitem}
Through this feature, the model gains enhanced semantic cues, which improve its ability to detect tokens with functional significance above visual presentation form. The feature precisely detects citation components, especially for their publication placement (\texttt{in}) as well as their authorship (\texttt{author}, \texttt{editor}, \texttt{translator}) and reference access methods (\texttt{url}, \texttt{arxiv}, \texttt{pubmed}).
Tokens that match the established keyword categories function as strong indicators for the field, but only some tokens have matching entries.

\subsubsection{Locator Feature}
The Locator function identifies persistent digital identifiers along with external resource pointers that function as tokens. The system contains the main groups of locators: DOIs, URLs, ISBNs, and PubMed IDs, along with additional academic citation locator types. Recognition of these tokens is vital because they most commonly occur at citation endings while holding semantic and structural differences from the rest of the fields.
The feature uses regular expression patterns to detect persistent digital identifiers, which also include commonly used external resource pointers:
\begin{compactitem}
\item The token detection system verifies terms which include \texttt{DOI}, \texttt{ISBN}, \texttt{URL}, \texttt{PMCID} and \texttt{PubMed}.
\item A Digital Object Identifier stands as \texttt{10.} followed by a numeric prefix and a suffix which combines as \texttt{10.1000/xyz123}.
\item The feature detects typical URI forms which start with \texttt{http://}, \texttt{https://}, or \texttt{ftp://} within web addresses.
\end{compactitem}
The feature returns \texttt{'T'} (true) when any defined patterns match within the analyzed token, indicating a potential locator. Otherwise, it returns \texttt{'F'} (false).
Example:
\begin{compactitem}
\item \texttt{https://doi.org/10.1007/s00799-018-0242-1} → \texttt{T}
\item \texttt{PMID: 12345678} → \texttt{T}
\item \texttt{Springer} → \texttt{F}
\end{compactitem}
Through this feature, the model detects tokens containing external source references and accesses information, thus enhancing its ability to accurately classify fields, particularly for digital and web-based references.

\subsubsection{Number Feature}
The number feature categorizes tokens based on whether and how there is numerical information. Numbers are essential in the middle of many bibliographic fields — e.g., years, volume numbers, ranges of pages, ISBNs, or identifiers — and understanding their format can help the model to decide the token's likely function in a citation. This feature applies a series of pattern-matching rules to translate all numeric tokens into a particular category. Some of the main categories include:
\begin{compactitem}
\item \textbf{volume:} Tokens that have the appearance of a volume and issue format, such as \texttt{12(3)} or \texttt{5:7}.
\item \textbf{isbn:} Strings that conform to the pattern of ISBN numbers, both starting with \texttt{978} and those starting with \texttt{979}.
\item \textbf{year:} Four-digit years from a reasonable range of history, such as \texttt{1998} or \texttt{2023}.
\item \textbf{quad, triple, double, single:} Tokens made up of 4, 3, 2, or 1 digits respectively. These simple forms typically represent years, page numbers, or brief identifiers.
\item \textbf{all:} Tokens made entirely of digits, but not in one of the other specialized forms.
\item \textbf{range:} Numerical ranges with hyphens, e.g., \texttt{123–145}, commonly page numbers.
\item \textbf{idnum:} Alphanumeric identifiers in which a number is prefixed with letters or codes, e.g., \texttt{ABC-123} or \texttt{ISSN2049}.
\item \textbf{ordinal:} Numerical and alphabetical combination tokens, e.g., \texttt{3rd}, \texttt{21st}, or \texttt{2a}, appearing infrequently in editions or titles.
\item \textbf{numeric:} Tokens having at least one digit but none of the preceding patterns.
\item \textbf{roman:} Roman numerals like \texttt{III}, \texttt{XIV}, or \texttt{iv}, appearing infrequently to number chapters, volumes, or appendices.
\end{compactitem}
Otherwise, if the token does not belong to any known numeric pattern, it gets labeled as \texttt{none}.

By classifying these fine-grained numeric types, the model gains a deeper sense of the organization of the reference, having the ability to distinguish between a publication year, an ISBN, and a volume/issue number, even when all appear to be numbers. This helps to improve the precision of field classification, especially for citation styles that vary in the way and where numbers appear.

\subsubsection{Position Feature}
The position feature holds the relative location of a token within the reference string. In citation parsing, when a token's position can powerfully predict its role, its position tends to reveal a great deal about its function. Authors' names tend to be toward the front, say, whereas publication dates, URLs, or page numbers tend to appear closer to the end.
This feature gives back one of the following values, depending on the token's position in the sequence:
\begin{compactitem}
\item \textbf{only:} If the token is the only token in the sequence
\item \textbf{first:} If the token is the first token in the sequence
\item \textbf{last:} If the token is the last token in the sequence
\item \textbf{A relative position value} (as an integer between 0 and 10) if the token is in the middle somewhere
\end{compactitem}
The relative position is calculated as a coarse-grained proportion: the index of the token divided by the number of tokens and adjusted by an absolute level of precision (for example, 0 to 10). As an example, a token in the middle of a sequence of 20 tokens would receive the value 5.

By exposing the model to this positional information, the feature helps the model learn to recognize in which portions of a reference string one will most likely discover specific types of information. This can be extremely helpful in free-form or otherwise variably styled references, where even the formatting cannot give sufficient indication for field separation.

\subsubsection{Punctuation Feature}
The punctuation feature recognizes the presence and type of punctuation in a token. Punctuation will generally play a structural role in citation strings, separating fields or designating formatting conventions. For example, colons will separate titles and subtitles, periods will designate abbreviations, and hyphens will denote ranges like page numbers or dates.
The feature looks at each token and labels it based on the punctuation that it contains:
\begin{compactitem}
\item \textbf{none:} The token does not have any punctuation.
\item \textbf{colon:} The token contains a colon (\texttt{:}), which is used for title or subtitle delineation.
\item \textbf{hyphen:} The token contains a hyphen or dash, which can indicate a range (i.e., \texttt{123–145}) or be part of a compound word.
\item \textbf{period:} The token contains a period (\texttt{.}), which can indicate abbreviations or sentence finality.
\item \textbf{amp:} The token has an ampersand (\texttt{\&}), often used to join author names (e.g., \texttt{Smith \& Johnson}).
\item \textbf{other:} The token has punctuation not in one of the above classes.
\end{compactitem}
By capturing these distinctions, the punctuation feature provides useful cues to token boundaries, field separators, and potential abbreviations — all of which are useful to the model in annotating different parts of a citation.

\subsubsection{Terminal Feature}
Terminal property identifies in which way a token is terminating, namely if it is terminating in punctuation, brackets, or quotes. In citations of bibliography, the punctuation a token is terminating in often signals the end of a field or the separation between units of meaning, e.g., the end of an author name, title, or publication date.
This feature looks at the trailing characters of the token and categorizes it as one of four based on the quality of the ending punctuation it possesses:
\begin{compactitem}
\item \textbf{strong:} The token contains a strong punctuation mark such as a period (\texttt{.}), closing parenthesis (\texttt{)}), or square bracket (\texttt{]}), possibly followed by a quotation mark. These tend to mark the end of a field or sentence.
\item \textbf{moderate:} The token is suffixed with a quotation or colon. It may be preceded by a lighter punctuation character at times. These may mark a transition, like the start of a subtitle or inline citation.
\item \textbf{weak:} The token is preceded by a lighter punctuation character like a comma, semicolon, hyphen, or exclamation mark. These mark continuation but can still segment parts of a field.
\item \textbf{none:} The token ends without any meaningful punctuation.
\end{compactitem}
For example:
\begin{compactitem}
\item \texttt{2023).} → \texttt{strong}
\item \texttt{"Chapter 2:} → \texttt{moderate}
\item \texttt{Vol. 5,} → \texttt{weak}
\item \texttt{Science} → \texttt{none}
\end{compactitem}
By maintaining these distinctions, the terminal feature helps the model to learn where fields most likely begin or end, especially in citation styles where such patterns are not absolutely dictated by format but instead are a function of punctuation. It plays a silent but important role in improving the accuracy of token labelling across citation formats.
\clearpage

\subsection*{Summary of Handcrafted Features}
The new hand-crafted features employed in this work are inspired by how reference parsing is addressed in AnyStyle~\cite{anystyle}, but are reworked and extended to more effectively meet the needs of modern neural models. They encode a dense array of linguistic, structural, and semantic cues — including orthographic features, token position, punctuation, character types, and semantic dictionaries. Each of these provides a different perspective on the reference string, and together they form a dense representation that allows accurate field labeling independent of citation style. Later, we evaluate the contribution of these features in isolation as well as in combination with embedding-based methods.
\renewcommand{\arraystretch}{1.3}
\begin{table}[ht]
\centering
\caption{Summary of handcrafted features used in the citation parsing model}
\label{tab:handcrafted-features}
\begin{tabular}{p{4cm}p{10cm}}
\toprule
\textbf{Feature} & \textbf{Description} \\
\midrule
\texttt{AffixFeature} & Extracts prefixes and suffixes from each token to detect morphological patterns and common abbreviations. \\
\texttt{BracketsFeature} & Identifies tokens enclosed in or adjacent to brackets like \texttt{()}, \texttt{[]}, or \texttt{<>}, often used for years or references. \\
\texttt{CanonicalFeature} & Produces a normalized lowercase version of the token without accents or formatting noise. \\
\texttt{CapsFeature} & Classifies tokens based on capitalization, e.g., all-caps, initials, or lowercase. \\
\texttt{CategoryFeature} & Returns the Unicode category of the first and last character, helping to detect punctuation, digits, or letters. \\
\texttt{DictionaryFeature} & Checks if a token exists in domain-specific dictionaries: names, publishers, journals, or places. \\
\texttt{KeywordFeature} & Matches tokens against keyword patterns to detect roles like editor, translator, journal, or date terms. \\
\texttt{LocatorFeature} & Detects persistent identifiers like DOIs, URLs, PubMed IDs, and ISBNs. \\
\texttt{NumberFeature} & Classifies numeric tokens (e.g., years, volumes, page ranges, Roman numerals) based on their structure. \\
\texttt{PositionFeature} & Encodes the position of a token within the reference string (first, last, middle, etc.). \\
\texttt{PunctuationFeature} & Identifies punctuation within a token (colon, hyphen, ampersand, etc.). \\
\texttt{TerminalFeature} & Examines how a token ends to infer punctuation strength and field boundaries. \\
\bottomrule
\end{tabular}
\end{table}
\renewcommand{\arraystretch}{1.0}



\clearpage

\subsection{Embedding-Based Features}
Embedding-based features provide dense, learned token representations through embedding semantic and contextual details. Contrary to handcrafted features that rely on surface patterns and professional knowledge, embeddings are learned in large bodies of text and may involve subtle language use, structural dependencies, and sense. In this paper, we experiment with two types of embeddings: Byte-Pair Encoding (BPE) embeddings and BERT-based contextual embeddings. These embeddings are used alone or blended with hand-crafted features to enhance the model's ability to label tokens appropriately in reference strings.

\subsubsection{Byte-Pair Encoding (BPE) Embeddings}
Byte-Pair Encoding (BPE) embeddings offer a tokenization-free, light, and multilingual way of subword representation. We incorporate in this paper pre-trained BPE embeddings released by the BPEmb project, which offers subword-level vector representations for 275 languages, including low-resource languages~\cite{bpemb}.
BPE is a data-driven compression algorithm that continuously merges the most frequent adjacent symbol pairs in a sequence. For example, in English, the most frequent pair, such as \texttt{t} and \texttt{h} might be merged into \texttt{th}, then further pairs such as \texttt{th} and \texttt{e} into \texttt{the}, depending on frequency. The number of merge operations the unit receives will determine its granularity as a subword unit, with fewer being more finely grained at the character level and more being units that are more like whole words~\cite{bpemb}.

BPEmb takes advantage of this idea to split untokenized, raw Wikipedia text into many languages and then learns embeddings across subword units using the GloVe algorithm~\cite{bpemb}~\cite{glove}. The method has the following significant advantages:
\begin{compactitem}
\item \textbf{No tokenization required,} especially suitable for languages without word boundaries (for example, Chinese, Japanese).
\item \textbf{Multilinguality,} with embeddings trained over a wide variety of languages, ranging from high-, medium-, and low-resource languages.
\item \textbf{Compact size,} outperforming other models like FastText in certain languages while using significantly less memory (e.g., 11MB for BPEmb vs. 6GB for FastText)~\cite{bpemb}~\cite{bojanowski-enriching}.
\end{compactitem}
On tasks of evaluation, such as fine-grained entity typing, BPEmb outperformed or matched both FastText and character-based models in several languages, including English, Chinese, and Tibetan~\cite{bpemb}. This makes BPE embeddings particularly valuable in low-resource settings or scenarios involving efficient memory usage.
The embeddings used here were selected based on their performance–dimensionality trade-offs and were fused as features in addition to handcrafted features. This allowed the model to leverage both learned semantic representations and clear, human-designed signals.


\subsubsection{BERT Embeddings}
BERT (Bidirectional Encoder Representations from Transformers) is a language representation model created by Devlin et al.~\cite{2019-bert} to obtain superior performance on downstream NLP tasks by providing deep, bidirectional contextual embeddings. Unlike earlier models such as GPT~\cite{gpt-2018} or ELMo~\cite{elmo}, BERT is pre-trained to encode both left and right context together at all layers, rather than using two separate models.
BERT uses a multi-layer Transformer encoder model, expanding on Vaswani et al.~\cite{attention-2017}. There are two main versions of the model:
\begin{compactitem}
\item\textbf{BERT\textsubscript{BASE}:} 12 layers, 768 hidden units, 12 attention heads, 110 million total parameters.
\item \textbf{BERT\textsubscript{LARGE}:} 24 layers, 1024 hidden units, 16 attention heads, 340 million parameters.
\end{compactitem}
To pre-train its embeddings, BERT relies on two significant unsupervised objectives:
\begin{compactitem}
\item \textbf{Masked Language Modeling (MLM):} 15\% of the tokens in each input sequence are randomly masked during training, and the model is trained to predict the original tokens based on the entire bidirectional context. This trains the model to capture deeper language patterns and dependencies.
\item \textbf{Next Sentence Prediction (NSP):} The model is trained to predict if sentence B follows sentence A, in order to enable it to learn sentence-level relations crucial for tasks like question answering and textual entailment.
\end{compactitem}
BERT's input representation is constructed by combining three types of embeddings:
\begin{compactitem}
\item \textbf{Token embeddings} based on WordPiece tokenization~\cite{wordpiece}.
\item \textbf{Segment embeddings} to differentiate between sentences in a pair.
\item \textbf{Positional embeddings} that convey token order.
\end{compactitem}
These three parts are summed up to form the final embedding of each token. All inputs begin with a special [CLS] classification token, and [SEP] tokens are used to delimit different sentences.
There are two major ways to use BERT:
\begin{compactitem}
\item \textbf{Fine-tuning:} The entire model is fine-tuned on a specific downstream task by adding a small output layer on top.
\item \textbf{Feature extraction:} BERT is used to generate contextual token embeddings, which are then used as input to another model, e.g., a sequence tagger.
\end{compactitem}

Benefits of BERT Embeddings:
\begin{compactitem}
\item They are deeply bidirectional, representing full context around each token.
\item They are pre-trained on large corpora, including the BooksCorpus and English Wikipedia, so they are robust and generally applicable.
\item They achieve state-of-the-art performance on a range of tasks, including sentence classification, named entity recognition, and question answering.
\end{compactitem}

\begin{figure}[ht]
    \centering
    \begin{tikzpicture}[
  every node/.style={font=\small},
  emb/.style={rectangle, minimum width=1.2cm, minimum height=0.8cm, draw, fill=gray!15},
  arrow/.style={thick, ->, >=Stealth},
  sum/.style={circle, draw, fill=blue!10, minimum size=0.6cm, inner sep=0pt}
]

% Tokens
\node[emb, fill=red!10] (cls) {[CLS]};
\node[emb, fill=red!10, right=0.2cm of cls] (the) {The};
\node[emb, fill=red!10, right=0.2cm of the] (dog) {dog};
\node[emb, fill=red!10, right=0.2cm of dog] (is) {is};
\node[emb, fill=red!10, right=0.2cm of is] (cute) {cute};
\node[emb, fill=red!10, right=0.2cm of cute] (sep1) {[SEP]};
\node[emb, fill=red!10, right=0.2cm of sep1] (it) {It};
\node[emb, fill=red!10, right=0.2cm of it] (likes) {likes};
\node[emb, fill=red!10, right=0.2cm of likes] (playing) {playing};
\node[emb, fill=red!10, right=0.2cm of playing] (sep2) {[SEP]};

% Embedding layers
\foreach \name/\above in {cls/cls, the/the, dog/dog, is/is, cute/cute, sep1/sep1, it/it, likes/likes, playing/playing, sep2/sep2} {
  \node[emb, fill=blue!5, below=1cm of \name] (tok-\name) {T\textsubscript{Emb}};
  \node[emb, fill=green!10, below=1cm of tok-\name] (seg-\name) {S\textsubscript{Emb}};
  \node[emb, fill=orange!10, below=1cm of seg-\name] (pos-\name) {P\textsubscript{Emb}};

  % Arrows from each embedding
%   \draw[arrow] (tok-\name.north) -- (sum-\name.south);
%   \draw[arrow] (seg-\name.north) -- (sum-\name.south);
%   \draw[arrow] (pos-\name.north) -- (sum-\name.south);
%   \draw[arrow] (sum-\name.north) -- (\name.south);
}

\node[sum, left=1.5cm of seg-cls] (sum) {$\sum$};
\draw[arrow] (sum) -- (tok-cls);
\draw[arrow] (sum) -- (seg-cls);
\draw[arrow] (sum) -- (pos-cls);

% Labels
\node[align=center, above=0.2cm of the, font=\bfseries] {BERT Input Representation};

\end{tikzpicture}

    \caption[BERT Input Representation]{The BERT input representation: each token is represented as the sum of its token, segment, and positional embeddings. Inspired by the design shown in~\cite{2019-bert}.}
    \label{fig:bert-input}
\end{figure}

\textbf{Linq-Embed-Mistral: A Modern BERT-style Encoder}

While the original BERT model has been an essential architecture for natural language processing, it has been the target of numerous improvements in terms of training efficiency, multilingual generalization, and embedding quality. As explained in the Hugging Face article about new variants of BERT, newer models such as MiniLMv2, E5, and Mistral-based encoders have comparatively much better performance on embedding tasks, particularly for retrieval, classification, and sentence similarity use cases.

In this work, rather than using the baseline BERT model~\cite{2019-bert}, we decided to use a modern BERT-style encoder from the Hugging Face leaderboard, specifically the Linq-Embed-Mistral model~\cite{linq}. The encoder is a fusion of the Mistral transformer backbone with simplified embedding tuning, offering dense high-quality representations well adapted to token-level tasks like reference parsing. The model was one of the top-performing multilingual text encoders and offered strong out-of-the-box performance in zero-shot or low-shot settings — a valuable property for parsing reference styles not directly seen at training time.
The Linq-Embed-Mistral model was used in the same application as BPE or standard BERT embeddings, generating subword-level context embeddings for each token in a reference string.
\newline
\newline
\textbf{DistilBERT: Lightweight Multilingual Embeddings}

In addition to modern transformer architectures, we also experimented with using lighter and faster versions of substitutes to make token embeddings. DistilBERT~\cite{sanh2019distilbert} is a light version of the original BERT base model~\cite{2019-bert}, which is trained to retain most of BERT's performance with 40\% less model size, 60\% increased inference speed, while preserving nearly 97\% of its language knowledge capabilities.

To conduct our experiments, we employed specifically the multilingual cased model of DistilBERT, by Hugging Face~\cite{distilbert-multilingual}. This is trained on the same data as mBERT (Multilingual BERT) and performs well over 100 languages. The multilingual cased DistilBERT provided compact, efficient representation that proved of immense value to handle citation data in many languages, preserving casing information in the process — essential to reference parsing since proper names, acronyms, and titles can significantly depend on capitalisation.
In this paper, the DistilBERT embeddings were generated at the token level similar to Linq-Embed-Mistral or BPE, allowing us to contrast directly the impact of model architecture size as well as representational quality on the reference parsing task.


\clearpage

\subsection{Feature Integration}
A key challenge in this work is integrating information from different levels of granularity: \textbf{subword-based embeddings} (e.g., BERT, BPEmb) and \textbf{word-level handcrafted features} (e.g., affix, capitalization, punctuation). This integration is handled differently depending on the model architecture.

\textbf{Neural Models (e.g., BiLSTM + CRF)}

In models that consume dense vector representations, each input token is represented by combining learned contextual embeddings with structured handcrafted features:
\begin{compactitem}
\item \textbf{Subword Embeddings:} Tokens are passed through pretrained models such as BERT or BPEmb. These models tokenize words into subword units internally (e.g., “information” may be split into "in", "\#\#formation"). Since handcrafted features operate at the word level, we align the two by averaging the embeddings of subwords corresponding to each original word token. This produces a single subword-aware vector per token.
\item \textbf{Handcrafted Features:} Each token also has categorical handcrafted features (e.g., prefix, caps type). Each of these is treated as a separate vocabulary and embedded using \texttt{nn.Embedding} layers in PyTorch. These embeddings are trained alongside the model.
\end{compactitem}
The two components are \textbf{aligned} in dimensionality: the subword embedding is passed through a BiLSTM layer, and the handcrafted feature embedding is projected via a linear layer to match the LSTM output’s dimensions. The two vectors are then added element-wise and passed to a CRF layer for structured prediction.

\textbf{Sparse CRF Models (e.g., python-crfsuite)}

In models like CRFsuite that do not support dense vector inputs, subword embeddings are converted to feature dictionaries. Each token's embedding is flattened and assigned feature names with numeric indices (e.g., \texttt{bert\_0\=0.123}). If two embedding sources are used (e.g., BERT and hand features), their features are merged using distinct prefixes. Handcrafted features, in this case, are also included as categorical string features.
Since CRFsuite works with string-based sparse features, there is no need for alignment between the two embedding sources—each word is treated as a single token, and the embedding representation of a word that contains multiple subwords is averaged beforehand during preprocessing.

This setup ensures that both contextual richness from deep embeddings and structural insight from handcrafted features are available to the model. Dense architectures learn the interaction between the two during training, while sparse CRFs receive both as complementary symbolic inputs.

